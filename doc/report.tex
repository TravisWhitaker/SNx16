\documentclass{article}
\usepackage[letterpaper]{geometry}
\usepackage{amsmath}
\usepackage{tikz}
\usepackage{listings}
\frenchspacing

\lstset{language=[x86masm]Assembler, basicstyle=\small}

\begin{document}

\title{SNx16 - 16 bit Constant Width Instruction Set Architecture}
%\subtitle{Assembly Language, Instruction Encoding, Control Logic,
%and Data Path Specification}
\author{Chris Ranc, Travis Whitaker}
\date{October 23\textsuperscript{rd} 2015}
\maketitle

\begin{abstract}
\emph{SNx16} a 16 bit instruction set architecture with constant instruction
width. This document formally specifies an assembly language grammar and
semanics for the ISA, the instruction encoding scheme, and the necessary control
logic and data path hardware for a simple non-pipelined implementation of the
architecture. \emph{SNx16} is a Harvard machine with separate 16 bit address
spaces for machine code and data.
\end{abstract}

%\thispagestyle{empty}
%\clearpage
%\setcounter{page}{1}

\section{Introduction and Rationale}

The design and specification of \emph{SNx16} is an exercise in the funadmental
principals of computer architecture. The architecture makes use of 16 bit
registers and memory addresses, and has separate memory address spaces for
program code and data. Program code is immutable at program runtime; only one
translation unit of object code may be loaded into the machine at once, and the
machine has no facility to bootstrap new object code on its own.

The architecture is minimal (implementing only eight instruction mnemonics) and has no facility for I/O, but is nonetheless useful for writing simple programs. To demonstrate this, a working assembler, linker, and bytecode interpreter are provided along with this specification.

\section{Assembly Language Grammar and Semantics}
\subsection{Registers, Flags, and Memory}
\subsection{Arithmetic Instructions}
\subsection{Data Movement Instructions}
\subsection{Control Flow Instructions}
\subsection{Example Program}

\section{Instruction Encoding}

\section{Control Logic and Data Path}

\section{\emph{snxi} Interpreter}

\end{document}
